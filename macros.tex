% för att enkelt formattera lösningar
\newenvironment{solution}[1]
    {\addcontentsline{toc}{subsubsection}{#1}{\Large\textbf{#1.}}\newline}
    {\newline}

\newcommand{\subsolution}[1]{{\large\textbf{#1)}\newline}}

% för att skapa "formler"
\newtheorem{formel}{Formel}

% för att skapa "definitioner"
\newtheorem{definition}{Definition}

% kommando för att skriva R^n
\newcommand{\Rn}[1]{\mathbb{R}^{#1}}

% environment för att enkelt skapa rutnät där man kan rita vektorer
\newenvironment{vectors2d}[4]
{
\begin{center}
\begin{tikzpicture}[scale=0.5]
  \draw[thin,gray!40] (#1,#2) grid (#3,#4);
  \draw[<->] (#1,0)--(#3,0) node[right]{$x$};
  \draw[<->] (0,#2)--(0,#4) node[above]{$y$};
}
{
\end{tikzpicture}
\end{center}
}
% Lägg in sånna här för att rita vektorer:
% \draw[line width=1pt,gray!40,-stealth](2,3)--(1,-2) node[anchor=south west]{$\vec{u}$};
% om man vill färga den:
% \draw[line width=1pt,blue,-stealth](2,3)--(1,-2) node[anchor=south west]{$\vec{u}$};


%% inkludera alla lösningar i en mapp 
%% uppgifter som saknas skrivs upp i table of contents med, fast som missing
\newcommand{\inkluderauppgifter}[2]{
\foreach \i in {1, ..., #2} {
    \edef\FileName{#1/\i}%
    \IfFileExists{\FileName}{%
       \subfile{\FileName}%
    }{%else
        {\addcontentsline{toc}{subsubsection}{\color{red} \i - lösning saknas}}%
    }%
}%
}




%% totalmatris
% https://tex.stackexchange.com/questions/2233/whats-the-best-way-make-an-augmented-coefficient-matrix
\makeatletter
\renewcommand*\env@matrix[1][*\c@MaxMatrixCols c]{%
  \hskip -\arraycolsep
  \let\@ifnextchar\new@ifnextchar
  \array{#1}}
\makeatother