\documentclass[../../main.tex]{subfiles}
\begin{document}
\begin{solution}{9}
En parameterform för ett plan består av
\begin{itemize}
    \item 1 punkt som ligger i planet.
    \item 2 vektorer som är parallella med planet.
\end{itemize}

Vi har tre punkter som ligger i planet. Man kan välja en av dessa som 'startpunkt' och sedan ta fram de två vektorerna genom subtraktion. Låt oss välja (1, 2, 0) som startpunkt, då ser det ut såhär

\begin{center}
\begin{tikzpicture}
% rita punkter
\tkzDefPoint(0,0){Start}
\tkzLabelPoint[left](Start){$(1, 2, 0)$}
\tkzDefPoint(5,2){A}
\tkzLabelPoint[right](A){$(2, 1, 1)$}
\tkzDefPoint(5,-2){B}
\tkzLabelPoint[right](B){$(0, -1, 5)$}
\tkzDrawSegment[gray!40](Start,A)
\tkzDrawSegment[gray!40](Start,B)

\foreach \n in {Start, A, B}
  \node at (\n)[circle,fill,inner sep=1.5pt]{};
\end{tikzpicture}
\end{center}

En vektor från (1, 2, 0) till (2, 1, 1) är
\[(2, 1, 1) - (1, 2, 0) = (1, -1, 1)\]

En vektor från (1, 2, 0) till (0, -1, 5) är
\[(0, -1, 5) - (1, 2, 0) = (-1, -3, 5)\]

En parameterframställning för planet är alltså
\[(x, y, z) = (1, 2, 0) + s \cdot (1, -1, 1) + t \cdot (-1, -3, 5) \text{ där } s, t \in R\]

\end{solution}
\end{document}