\documentclass[../../main.tex]{subfiles}
\begin{document}
\begin{solution}{17}
I den här uppgiften vill vi använda båda varianterna av skalärprodukten (formel \ref{skalarprod}). Börja med att observera att skalärprodukten mellan en vektor $\Vec{v} = (x, y, z)$ och x-, y- och z-axeln (med den första varianten av skalärprodukten) är
\begin{align*}
(x, y, z) \cdot (1, 0, 0) = x\\
(x, y, z) \cdot (0, 1, 0) = y\\
(x, y, z) \cdot (0, 0, 1) = z\\
\end{align*}

Nu använder den andra varianten av skalärprodukten för att räkna ut $x, y, z$. Den vektor vi söker har längd 2, $|\Vec{v}| = 1$, och enhetsvektorn för x-, y- och z-axeln har längd ett.

Börjar med x-axeln
\[
2 \cdot 1 \cdot cos\frac{\pi}{3} = 2 \cdot \frac{1}{2} = 1
\]

Sedan y-axeln
\[2 \cdot 1 \cdot cos\frac{3\pi}{4} = 2 \cdot \frac{-1}{\sqrt{2}} = -\sqrt{2}\]

Och till sist z-axeln
\[2 \cdot 1 \cdot cos\frac{2\pi}{3} = 2 \cdot \frac{-1}{2} = -1\]

Eftersom att båda varianterna skalärprodukterna beräknar samma sak får vi $x = 1, y = -\sqrt{2}, z=-1$, vektorn vi söker är alltså $(1, -\sqrt{2}, -1)$.

\end{solution}
\end{document}