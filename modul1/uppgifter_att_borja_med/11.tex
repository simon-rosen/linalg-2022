\documentclass[../../main.tex]{subfiles}
\begin{document}
\begin{solution}{11} 

Följande behövs för parameterframställningen av en linje som ligger i planet
\begin{itemize}
    \item En punkt i planet.
    \item En vektor som är parallell med planet.
\end{itemize}

$x + y + z = 1$ uppfylls t.ex. av punkten (1, 0, 0), så här har vi en punkt som ligger i planet.

En vektor som är parallell med planet ska vara ortogonal med planets normalvektor. Vi ser från planets ekvation att denna är $\Vec{n} = (1, 1, 1)$. Använder skalärprodukten (formel \ref{skalarprod}) för att ta fram en vektor $\Vec{v}$ som är ortogonal mot $\Vec{n}$

\[\Vec{n} \cdot \Vec{v} = (1, 1, 1) \cdot (v_1, v_2, v_3) =
v_1 + v_2 + v_3 = 0\]

Detta uppfylls t.ex. av $v_1 = -1, v_2 = 1, v_3 = 0$. En parameterframställning av linjen är alltså

\[(x, y, z) = (1, 0, 0) + t \cdot (-1, 1, 0), t\in R\]

\end{solution}
\end{document}