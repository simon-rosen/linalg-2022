\documentclass[../../main.tex]{subfiles}
\begin{document}
\begin{solution}{18}

\subsolution{a}
Punkten ligger i planet om den uppfyller planets ekvation. Testar

\begin{align*}
z = 19 - 2x - 3y\\
(12) = 19 - 2(2) -3(1) = 19 - 4 - 3 = 12
\end{align*}

Punkten uppfyller planets ekvation och ligger alltså i planet.

\subsolution{b}
Jag börjar med att skriva om planets ekvation på en mer bekant form
\begin{align*}
z = 19 - 2x - 3y\\
2x + 3y + z = 19
\end{align*}

Enligt formel \ref{normalplan} har planet normalvektorn $\Vec{n} = (2, 3, 1)$.

\subsolution{c}

\end{solution}
\end{document}