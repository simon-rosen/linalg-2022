\documentclass[../../main.tex]{subfiles}
\begin{document}
\begin{solution}{8}
Den vektor som är ortogonal mot planet kallas för planets normalvektor. Planet har normalvektor $n = (1, 2, 3)$, detta fås genom att man kollar på koefficienterna till $x, y, z$ i planets ekvation $x+2y+3z=5$. Eftersom att vi vet att linjen kommer gå genom origo kan vi välja parameterformen
\[L = (0, 0, 0) + t \cdot (1, 2, 3) = t \cdot (1, 2, 3), t \in R\]
\end{solution}
\end{document}