\documentclass[../../main.tex]{subfiles}
\begin{document}
\begin{solution}{5}

Vi söker de enhetsvektorer som är ortogonala mot vektorerna $(2,-6,-3)$ \textbf{och}  $(4,3,-1)$. Kom ihåg att enhetsvektorer är vektorer av längd ett och att ortogonalitet betyder vinkelrät. Två vektorer är ortogonala om vinkeln mellan dom är $90^\circ=\frac{\pi}{2}$ radiener. 

Vi döper våra kända vektorer till $\overline{v}=(2,-6,-3)$ och $\overline{u}=(4,3,-1)$. Sedan definierar vi, till att börja med, en av våra sökta enhetsvektor $\overline{x}=(x,y,z)$. Med dessa definierade kan vi nu skriva upp våra villkor med matematisk notation. 

\bigskip

Att $\overline{x}$ är en enhetsvektor ger oss att $|\overline{x}|=1$  vilket vi utvecklar till $\sqrt{x^2+y^2+z^2}=1$. 
Sedan har vi villkoren att $\overline{x}$ är ortogonal mot $\overline{v} \textbf{ och } \overline{u}$ vilket ger oss följande ekvationer 

\begin{align*}
    \overline{x}\cdot\overline{v}=|\overline{x}|\cdot|\overline{v}|\cdot\cos{\frac{\pi}{2}} \\
    \overline{x}\cdot\overline{u}=|\overline{x}|\cdot|\overline{u}|\cdot \cos{\frac{\pi}{2}} \\
\end{align*}

Vilka vi kan förenkla genom att använda oss av skalärprodukter och villkoret att $\frac{\pi}{2}=0$ till

\begin{align*}
     (x,y,z)\cdot(2,-6,-3)=0  \\
    (x,y,z)\cdot(4,3,-1)=0,
\end{align*}

och vidare utveckla till

\begin{align*}
     2x-6y-3z=0  \\
    4x+3y-z=0.
\end{align*}

Vi söker nu de tre variablerna $x,y,z$ som löser ekvationssystemet från våra tre villkor.
  
\begin{align*}
    \begin{cases} 
     2x-6y-3z&=0  \\
    4x+3y-z&=0.   \\
    \sqrt{x^2+y^2+z^2}&=1.
    \end{cases}
\end{align*}

Viktigt att notera att vi inte har ett linjärt ekvationssystem då den tredje ekvationen har variabler av grad 2. Vi börjar därför med att lösa de två linjära ekvationerna.

\bigskip

Multiplicerar vi den övre ekvationen med $-2$ och adderar till den undre ekvationen får vi följande
\begin{align*}
    \begin{cases} 
     2x-6y-3z&=0  \\
    4x+3y-z+(-2)\cdot(2x-6y-3z)&=0 + (-2\cdot0) 
    \end{cases}
\end{align*}

vilket vi förenklar till

\begin{align*}
    \begin{cases} 
     2x-6y-3z&=0  \\
    15y+5z&=0. 
    \end{cases}
\end{align*}

Vi delar sedan den övre ekvationen med $2$ och den undre ekvationen med $15$ och får

\begin{align*}
    \begin{cases} 
     x-3y-\frac{3}{2}z&=0  \\
    y+\frac{1}{3}z&=0 
    \end{cases}
\end{align*}

sedan uttrycker vi den undre ekvationen i termer av $z$

\begin{align*}
    \begin{cases} 
     x-3y-\frac{3}{2}z&=0  \\
    y&=-\frac{1}{3}z 
    \end{cases}
\end{align*}

och upptäcker att genom insättning av den undre ekvationen i den övre ekvationen får vi att

$$x-3y-\frac{3}{2}z=x-3\cdot(-\frac{1}{3}z) -\frac{3}{2}z=x+z-\frac{3}{2}z=x-\frac{1}{2}z.$$

vilket då ger oss ekvationssystemet 

\begin{align*}
    \begin{cases} 
     x=\frac{1}{2}z\\
    y=-\frac{1}{3}z.
    \end{cases}
\end{align*}

Eftersom vi nu uttryck $x$ och $y$ i termer av $z$ kan vi sätta in dess i den sista ekvationen. 
Detta ger oss att 

$$\sqrt{x^2+y^2+z^2}=\sqrt{(\frac{z}{2})^2+(-\frac{z}{3})^2+z^2}=$$


\bigskip
\textbf{Alternativ lösning med kryssprodukt}

Det här är ett typiskt tillfälle där kryssprodukten (formel \ref{kryssprod}) är användbar, eftersom att den ger just en vektor som är ortogonal mot de vektorer man tar kryssprodukten av. Jag tar fram den direkt med hjälp av formel \ref{kryssprod}
\[(2, -6, -3)\times (4, 3, -1) = (6 - (-9), -12 - (-2), 6 - (-24)) = (15, -10, 30)\]

Längden av den här vectorn tas fram med \ref{veclength} och är \(\sqrt{15^2 + (-10)^2 + 30^2} = \sqrt{1225}\). En av de enhetsvektorer vi söker är alltså enligt formel \ref{unitvec} 
\[\frac{1}{\sqrt{1225}}\cdot(15, -10, 30)\].
Den andra pekar åt motsatt håll och är alltså 
\[\frac{-1}{\sqrt{1225}}\cdot(15, -10, 30)\]

Man skulle kunna förenkla ytterligare om man vill för $\sqrt{1225} = 35$ (vilket inte alls är uppenbart om man saknar miniräknare)
\[\frac{1}{\sqrt{1225}}\cdot(15, -10, 30) = \frac{1}{35}\cdot(15, -10. 30) = \frac{1}{7}\cdot(3, -2, 6)\]
\end{solution}
\end{document}