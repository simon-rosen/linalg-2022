\documentclass[../../main.tex]{subfiles}
\begin{document}
\begin{solution}{6}

Uppgifter där man ska visa saker kan vara lite svåra att börja med, men ofta kan man börja testa sig fram med hjälp av den info som ges.

Här handlar det om ortogonalitet så då skulle man kunna börja testa att använda skalärprodukten (formel \ref{skalarprod}), eftersom att den ska vara 0 om två vektorer är ortogonala

\[
(\Vec{u} + \Vec{v}) \cdot (\Vec{u} - \Vec{v}) = (u_1 + v_1) \cdot (u_1 - v_1) + (u_2 + v_2) \cdot (u_2 - v_2) + ... + (u_n + v_n) \cdot (u_n - v_n) =
\]

Notera att varje term i högerledet går att utveckla med konjugatregeln. Då får vi 

\[
= (u_1^2 - v_1^2) + (u_2^2 - v_2^2) + ... +  (u_n^2 - v_n^2) =
\]

Uppgiften handlar ju också om att vektorerna ska ha samma längd och detta börjar ju likna formeln för vektorers längd (formel \ref{veclength}) lite. Jag skriver om den såhär

\begin{multiline}
= (u_1^2 + u_2^2 + ... + u_n^2) - (v_1^2 + v_2^2 + ... + v_n^2) =
\sqrt{(u_1^2 + u_2^2 + ... + u_n^2)} - \sqrt{(v_1^2 + v_2^2 + ... + v_n^2)} = |\Vec{u}|^2 - |\Vec{v}|^2 = 0 \\
\iff |\Vec{u}|^2 = |\Vec{v}|^2 \\
\iff |\Vec{u}| = \pm |\Vec{v}|
\end{multiline}

En vektors längd är ju positiv så \(|\Vec{u}| = |\Vec{v}|\)

\end{solution}
\end{document}