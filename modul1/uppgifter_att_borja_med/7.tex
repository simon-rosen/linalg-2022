\documentclass[../../main.tex]{subfiles}
\begin{document}
\begin{solution}{7}
\subsolution{a}
Vi vill skapa en parameterframställning med formel \ref{parameterlinje}. En vektor som går från (1, 0, -1) till (2, 3, -5) är
\[(2, 3, -5) - (1, 0, -1) = (1, 3, -4)\]
En möjlig parameterframställning är alltså 
\[L = (1, 0, -1) + t \cdot (1, 3, -4)\]
Där $t \in R$

\subsolution{b}
Det här är enkelt eftersom att vi redan har vektorn som är parallell med linjen $L$
\[(0, 0, 0) + t \cdot (1, 3, -4), t \in R\]

\end{solution}
\end{document}