\documentclass[../../main.tex]{subfiles}
\begin{document}
\begin{solution}{13}

Sånna här uppgifter kommer vara mycket enklare att lösa senare i kursen när ni har gått igenom gausseliminering. Men man kan tänka såhär också: Om $L_1$ och $L_2$ skär varandra kommer deras ekvationer att ge samma resultat vid den punkten

\[t \cdot (-1, 0, 2) = (1, 2, 2) + s \cdot (-1, 2, 0) \text{ där } s, t \in R\]

om man låter $t = 1, s = 2$ får man $(-1, 0, 2) = (1, 2, 2) + (-2, 4, 0) = (-1, 6, 2)$. Detta är ju sant för det första elementet i vektorerna men inte för element två och tre. Det går helt enkelt inte att uppfylla $VL = HL$ och en lösning saknas alltså, därför skär inte linjerna heller varandra.

\end{solution}
\end{document}