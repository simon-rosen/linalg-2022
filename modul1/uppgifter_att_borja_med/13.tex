\documentclass[../../main.tex]{subfiles}
\begin{document}
\begin{solution}{13}

Sånna här uppgifter kommer vara mycket enklare att lösa senare i kursen när ni har gått igenom gausseliminering. Men man kan tänka såhär också: Om $L_1$ och $L_2$ skär varandra kommer deras ekvationer att ge samma resultat vid den punkten

\[t \cdot (-1, 0, 2) = (1, 2, 2) + s \cdot (-1, 2, 0) \text{ där } s, t \in R\]

om man låter $t = 1, s = 2$ får man $(-1, 0, 2) = (1, 2, 2) + (-2, 4, 0) = (-1, 6, 2)$. Detta är ju sant för det första elementet i vektorerna men inte för element två och tre. Det går helt enkelt inte att uppfylla $VL = HL$ för några värden på $s$ och $t$ (man förstår nog detta enklast om man försöker göra så att $VL = HL$ en stund) så en lösning saknas, därför skär inte linjerna heller varandra.

Den här lösningen känns inte så bra för bara för att man själv inte hittar något $s,t$ så att $HL=VL$ betyder ju inte att det inte finns. Lösningen nedan med gausseliminering gör åtminstonne mig mer säker på att det inte finns någon skärningspunkt.

\textbf{Alternativ lösning med gausseliminering}

En skärningspunkt mellan $L_1$ och $L_2$ inträffar när deras ekvationer är lika med varandra

\begin{align*}
    t(-1, 0, 2) = (1, 2, 2) + s(-1, 2, 0)\\
    t(-1, 0, 2) - s(-1, 2, 0)= (1, 2, 2)\\
    (-t, 0, 2t) + (s, -2s, 0)=(1, 2, 2)\\
    (-t+s, -2s, 2t) = (1, 2, 2)
\end{align*}

Det här ger ett ekvationssystem med tre ekvationer

\begin{align*}
    \begin{cases} 
        -t+s&=1  \\
        -2s &= 2 \\
        2t &= 2
    \end{cases}
\end{align*}

Det här ekvationssystemet kan beskrivas med totalmatrisen, och sedan lösas med gausseliminering


$$
\begin{pmatrix}[cc|c]
  -1 & 1 & 1\\
  0 & -2 & 2\\
  2 & 0 & 2\\
\end{pmatrix} 
\sim 
\begin{pmatrix}[c]
  r_1\\
  -\frac{r_2}{2}\\
  \frac{r_3}{2}\\
\end{pmatrix}
\sim 
\begin{pmatrix}[cc|c]
  -1 & 1 & 1\\
  0 & 1 & -1\\
  1 & 0 & 1\\
\end{pmatrix} 
\sim 
\begin{pmatrix}[c]
  r_3\\
  r_2}\\
  r_1 + r_3 - r_2\\
\end{pmatrix}
\sim 
\begin{pmatrix}[cc|c]
  1 & 0 & 1\\
  0 & 1 & -1\\
  0 & 0 & 3\\
\end{pmatrix} 
$$

Den här resultatet av gausselimineringen betyder att

\begin{align*}
    \begin{cases} 
        t&=1  \\
        s = -1 \\
        0 = 3 \textbf{ (Det här är omöjligt)}
    \end{cases}
\end{align*}

Att vi har fått fram som en del av lösningen att $0 = 3$ betyder att ekvationssystemet saknar lösning, vilket i sin tur betyder att linjerna $L_1$ och $L_2$ inte skär varandra.

\end{solution}
\end{document}