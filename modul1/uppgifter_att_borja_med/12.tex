\documentclass[../../main.tex]{subfiles}
\begin{document}
\begin{solution}{12} 

Den här uppgiften bygger vidare på uppgift 11 så se till att göra den först.

En parameterframställning av en linje som inte skär planet ska innehålla
\begin{itemize}
    \item En punkt som inte ligger i planet.
    \item En vektor som är parallell med planet (det här är viktigt för annars kommer den skära planet någon gång).
\end{itemize}

Vi har en vektor $\vec{v} = (-1, 1, 0)$ som är parallell med planet från uppgift 11. Det enda vi behöver nu är en punkt som inte ligger i planet. En sådan punkt ska uppfylla $x + y + z \neq 1$. Vi testar med origo (0, 0, 0)

\[0 + 0 + 0 = 0 \neq 1\]

Nu har vi allt som krävs

\[(x, y, z) = (0, 0, 0) + t \cdot (-1, 1, 0) = t \cdot (-1, 1, 0), t \in R\]

\end{solution}
\end{document}