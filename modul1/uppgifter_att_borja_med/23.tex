\documentclass[../../main.tex]{subfiles}
\begin{document}
\begin{solution}{23}

Vi kallar $(1, 1) = \vec{u}$. Knepet för att lösa den här uppgiften är att använda projektionen av $\vec{v}$ på $\vec{u}$, och sedan skriva $\vec{v}$ som en summa av denna projektion och en vektor som är vinkelrät mot denna. Det kan visualiseras såhär

\begin{vectors2d}{0}{0}{5}{5}
    \draw[line width=1pt,-stealth](0,0)--(2,5) node[anchor=south east]{$\vec{v}$};
    \draw[line width=1pt,-stealth](0,0)--(1,1) node[anchor=north west]{$\vec{u}$};
    \draw[line width=1pt,gray!40,-stealth](0,0)--(3.5,3.5) node[anchor=north west]{$\vec{proj_{\vec{u}}(\vec{v})}$};
    \draw[line width=1pt,gray!40,-stealth](3.5,3.5)--(2,5) node[anchor=south west]{$\vec{v} - \vec{proj_{\vec{u}}(\vec{v})}$};
\end{vectors2d}

Vi beräknar projektionen (formel  \ref{projektionsformeln})

$$proj_{\vec{u}}(\vec{v}) = \frac{\vec{u}\cdot\vec{v}}{|\vec{u}|^2}\cdot\vec{u} = \frac{2+5}{2}\cdot (1, 1) = \frac{7}{2}(1, 1) = (\frac{7}{2}, \frac{7}{2})$$

Nu kan vi beräkna den vektor som är ortogonal genom att ta $\vec{v} - proj_{\vec{u}}(\vec{v})$

$$\vec{v} - proj_{\vec{u}}(\vec{v}) = (2, 5) - (\frac{7}{2}, \frac{7}{2}) = (-\frac{3}{2}, \frac{3}{2})$$

$\vec{v}$ kan alttså skrivas som
$$\vec{v}=(2, 5) = (\frac{7}{2}, \frac{7}{2}) + (-\frac{3}{2}, \frac{3}{2})$$

\end{solution}
\end{document}