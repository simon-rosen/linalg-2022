\documentclass[../main.tex]{subfiles}
\begin{document}

Vektorerna och punkterna i dessa exempel är tre dimensioner ($\Rn{3}$). Men formlerna fungerar på liknande sätt för en godtycklig dimension ($\Rn{n}$).

%% vektoraddition
\begin{formel}{(vektoraddition)}\\
\label{vecadd}
Två vektorer $\vec{u} = (u_1, u_2, u_3)$ och $\vec{v} = (v_1, v_2, v_3)$ adderas såhär:
\[
\vec{u} + \vec{v} = (u_1 + v_1, u_2 + v_2, u_3 + v_3)
\]

Kom ihåg att $\vec{u} - \vec{v} = \vec{u} + (-1)\cdot\vec{v}$
\end{formel}


%% vektormultiplikation med skalär
\begin{formel}{(vektormultiplikation med ett tal)}\\
\label{vecscale}
Ett tal $t$ och en vektor $\vec{u} = (u_1, u_2, u_3)$ multipliceras såhär:
\[
t \cdot \vec{u} = (t \cdot u_1, t \cdot u_2, t \cdot u_3)
\]
\end{formel}



%% längden av en vektor
\begin{formel}
\label{veclength}
Längden $|\vec{v}|$ av en vektor $\vec{v} = (x, y, z)$ fås genom:
\[
|\vec{v}| = \sqrt{x^2+y^2+z^2}
\]

\end{formel}


%% beräkna enhetsvektorn
\begin{formel}
\label{unitvec}
En enhetsvektor $\vec{e}_\vec{v}$ i samma riktning som en vektor $\vec{v}$ fås genom:
\[
    \vec{e}_{\vec{v}} = \frac{1}{|\vec{v}|} \cdot \vec{v} 
\]

Man skalar alltså om vektorn så att den får längd ett.
\end{formel}


%% skalärprodukten
\begin{formel}{(skalärprodukt)}
\label{skalarprod}
Skalärprodukten av två vektorer $\vec{u} = (u_1, u_2, u_3)$ och $\vec{v} = (v_1, v_2, v_3)$ är:
\[
\vec{u} \cdot \vec{v} = u_1 \cdot v_1 + u_2 \cdot v_2 + u_3 \cdot v_3
\]

En annan formel som kan användas är: 
\[
\vec{u} \cdot \vec{v} = |\vec{u}||\vec{v}|sin(\alpha)
\]
Där $\alpha$ är vinkeln mellan vektorerna $\vec{u}$ och $\vec{v}$. Denna formel är särskilt användbar om man vill räkna ut vinkeln mellan två vektorer.

Skalärprodukten av två vektorer är ett tal.
\end{formel}


%% kryssprodukten
\begin{formel}{(kryssprodukt)}
\label{kryssprod} 
Kryssprodukten av två vektorer $\vec{u} = (u_1, u_2, u_3)$ och $\vec{v} = (v_1, v_2, v_3)$ är
\[\vec{u}\times \vec{v} = (u_2\cdot v_3 - u_3\cdot v_2, u_2\cdot v_3 - u_3\cdot v_2, u_2\cdot v_3 - u_3\cdot v_2)\]

En annan formel som kan användas för att beräkna kryssprodukten är
\[\vec{u}\times \vec{v} = |\vec{u}||\vec{v}|sin(\alpha)\]
Där $\alpha$ är vinkeln mellan $\vec{u}$ och $\vec{v}$.

En viktig egenskap kryssprodukten $\vec{u}\times\vec{v}$ är att \underline{den är en vektor} som är ortogonal mot båda vektorerna $\vec{u}$ och $\vec{v}$.
\end{formel}


%% parameterframställning av linje
\begin{formel}
\label{parameterlinje} 
En linje kan beskrivas med en parameterframställning om man har en punkt $P = (p_1, p_2, p_3)$ och en vektor $\vec{v}=(v_1, v_2, v_3)$ som pekar i linjens riktning (som man skalar om med ett tal $t \in R$ för att kunna komma till alla punkter på linjen)
\[L = P + t \cdot \vec{v} = (p_1, p_2, p_3) + t\cdot (v_1, v_2, v_3)\]
\end{formel}






\end{document}