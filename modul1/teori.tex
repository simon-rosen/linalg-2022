\documentclass[../main.tex]{subfiles}
\begin{document}

Vektorerna och punkterna i dessa exempel är tre dimensioner ($\Rn{3}$). Men formlerna fungerar på liknande sätt för en godtycklig dimension ($\Rn{n}$).

%% vektoraddition
\begin{formel}{(vektoraddition)}\\
\label{vecadd}
Två vektorer $\vec{u} = (u_1, u_2, u_3)$ och $\vec{v} = (v_1, v_2, v_3)$ adderas såhär:
\[
\vec{u} + \vec{v} = (u_1 + v_1, u_2 + v_2, u_3 + v_3)
\]

Kom ihåg att $\vec{u} - \vec{v} = \vec{u} + (-1)\cdot\vec{v}$
\end{formel}


%% vektormultiplikation med skalär
\begin{formel}{(vektormultiplikation med ett tal)}\\
\label{vecscale}
Ett tal $t$ och en vektor $\vec{u} = (u_1, u_2, u_3)$ multipliceras såhär:
\[
t \cdot \vec{u} = (t \cdot u_1, t \cdot u_2, t \cdot u_3)
\]
\end{formel}



%% längden av en vektor
\begin{formel}
\label{veclength}
Längden $|\vec{v}|$ av en vektor $\vec{v} = (x, y, z)$ fås genom:
\[
|\vec{v}| = \sqrt{x^2+y^2+z^2}
\]

\end{formel}


%% beräkna enhetsvektorn
\begin{formel}
\label{unitvec}
En enhetsvektor $\vec{e}_\vec{v}$ i samma riktning som en vektor $\vec{v}$ fås genom:
\[
    \vec{e}_{\vec{v}} = \frac{1}{|\vec{v}|} \cdot \vec{v} 
\]

Man skalar alltså om vektorn så att den får längd ett.
\end{formel}


%% vektor mellan två punkter
\begin{formel}
\label{vecbetweendots}
En vektor som går från punkten $P = (p_1, p_2, p_3)$ till punkten $Q = (q_1, q_2, q_3)$ fås genom
$$\vec{OQ} - \vec{OP} = (q_1, q_2, q_3) - (p_1, p_2, p_3)$$

Notera att $\vec{OP}$ och $\vec{OQ}$ är punkternas ortsvektorer (vektorer från origo till punkten), för tekniskt sätt kan man inte addera två punkter.
\end{formel}


%% skalärprodukten
\begin{formel}{(skalärprodukt)}
\label{skalarprod}
Skalärprodukten av två vektorer $\vec{u} = (u_1, u_2, u_3)$ och $\vec{v} = (v_1, v_2, v_3)$ är:
\[
\vec{u} \cdot \vec{v} = u_1 \cdot v_1 + u_2 \cdot v_2 + u_3 \cdot v_3
\]

En annan formel som kan användas är: 
\[
\vec{u} \cdot \vec{v} = |\vec{u}||\vec{v}|cos(\alpha)
\]
Där $\alpha$ är vinkeln mellan vektorerna $\vec{u}$ och $\vec{v}$. Denna formel är särskilt användbar om man vill räkna ut vinkeln mellan två vektorer.

Skalärprodukten av två vektorer \underline{är ett tal}.
\end{formel}


%% parameterframställning av linje
\begin{formel}
\label{parameterlinje} 
En linje kan beskrivas med en parameterframställning om man har en punkt $P = (p_1, p_2, p_3)$ och en vektor $\vec{v}=(v_1, v_2, v_3)$ som pekar i linjens riktning (som man skalar om med ett tal $t \in R$ för att kunna komma till alla punkter på linjen)
\[(x, y, z) = P + t \cdot \vec{v} = (p_1, p_2, p_3) + t\cdot (v_1, v_2, v_3)\]
\end{formel}


%% parameterframställning av plan
\begin{formel}
\label{parameterplan} 
Ett plan kan beskrivas med en parameterframställning om man har en punkt $P = (p_1, p_2, p_3)$ samt två vektor $\vec{v}=(v_1, v_2, v_3)$ och $\vec{u}=(u_1, u_2, u_3)$ som är parallella med planet (om man skalar dessa vektorer med två tal $s, t \in R$ ska man kunna komma till alla punkter i planet)
\[(x, y, z) = P + s \cdot \vec{u} + t \cdot \vec{v} = (p_1, p_2, p_3) + s\cdot (u_1, u_2, u_3) + t \cdot (v_1, v_2, v_3)\]
\end{formel}


%% normalvektor till plan
\begin{formel}
\label{normalplan} 
Ett plan som beskrivs med en ekvation $ax + by + cz = d$ har en normalvektor $\vec{n} = (a, b, c)$ som har en rät vinkel till planet.
\end{formel}


\end{document}