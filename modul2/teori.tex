\documentclass[../main.tex]{subfiles}
\begin{document}
Här används främst 3x3 matriser, för andra dimensioner fungerar det på samma sätt.

%% gauss-jordans metod
\subsubsection{Gauss-Jordans metod}
\label{gaussjordan}
Gauss-Jordans metod går ut på att man steg för steg gör om en matris tills den har formen
$$
\begin{pmatrix}[ccc|c]
  1 & 0 & 0 & a\\
  0 & 1 & 0 & b\\
  0 & 0 & 1 & c
\end{pmatrix} 
$$

Man kan tolka detta som lösningen till ett ekvationssystem: $x = a, y = b, z = c$.

De elementära radoperationerna (som man får använda för att förändra matrisen) är:
\begin{itemize}
    \item Man får byta plats på rader i matrisen.
    \item Man får multiplicera varje rad med en skalär.
    \item Man får addera en rad till en annan (det är tilllåtet att addeda en rad $\cdot$ ett tal till en annan rad).
\end{itemize}

Om man skulle få en matris som ser ut typ såhär
$$
\begin{pmatrix}[ccc|c]
  1 & 0 & 0 & a\\
  0 & 1 & 0 & b\\
  0 & 0 & 0 & 1
\end{pmatrix} 
$$
är systemet \underline{inkonsistent} (olösbart), eftersom att det betyder att $0\cdot x + 0\cdot y + 0\cdot z = 1$ vilket är omöjligt.

Om man skulle få typ
$$
\begin{pmatrix}[ccc|c]
  1 & 0 & 0 & a\\
  0 & 1 & 1 & b\\
  0 & 0 & 0 & 0
\end{pmatrix} 
$$

d.v.s att det finns så kallade \underline{fria variabler}, betyder det att det inte finns en unik lösning till systemet utan det kommer finnas oändligt många lösningar. I det här exemplet är lösningen en linje i $\Rn{3}$.

\subsubsection{Homogena ekvationssystem}
Ett \underline{Homogent ekvationssystem} är ett ekvationssystem där alla ekvationer har $HL = 0$. T.ex

$$
\begin{cases}
x + y + z = 0\\
x + 2y +2 z = 0
\end{cases}
$$

Ett sådant system kan alltid lösas med den \underline{triviala lösningen} där man sätter $x = y = z = 0$.

\subsubsection{Matriser}
\begin{definition}
\label{matrisrang}
En matris \underline{rang} kan får genom att man räknar antalet ledande ettor i den matris man får genom gausseliminering.

Exempelvis har vi
$$A = \begin{pmatrix}
1 & 0 & 1\\
0 & 1 & 1\\
0 & 0 & 0
\end{pmatrix}$$
rang(A) = 2
\end{definition}

\end{document}