\documentclass[../../main.tex]{subfiles}
\begin{document}
\begin{solution}{13}

Vi använder gausseliminering på matrisen $A$

$$
A = \begin{pmatrix}[cccc]
1 & 2 & 3 & 4\\
2 & 3 & 4 & 5\\
1 & 1 & 1 & 2
\end{pmatrix}
\sim 
\begin{pmatrix}[c]
r_1\\
r_2 - 2\cdot r_1\\
r_3 - r_1
\end{pmatrix}
\sim 
\begin{pmatrix}[cccc]
1 & 2 & 3 & 4\\
0 & -1 & -2 & -3\\
0 & -1 & -2 & -2
\end{pmatrix}
\sim 
\begin{pmatrix}[c]
r_1 + 2\cdot r_2\\
-1\cdot r_2\\
r_3 - r_2
\end{pmatrix}
\sim
$$
% ny rad
$$
\begin{pmatrix}[cccc]
1 & 0 & -1 & -2\\
0 & 1 & 2 & 3\\
0 & 0 & 0 & 1
\end{pmatrix}
\sim 
\begin{pmatrix}[c]
r_1 + 2\cdot r_3\\
r_2 - 3\cdot r_3\\
r_3
\end{pmatrix}
\sim
\begin{pmatrix}[cccc]
1 & 0 & -1 & 0\\
0 & 1 & 2 & 0\\
0 & 0 & 0 & 1
\end{pmatrix}
$$

Det här är en matris på reducerad trappstegsform: Vi har bara ledande ettor, och de tal som är över/under ettorna är alla nollor. Man kan utföra gausselimineringen på olika sätt men man kommer alltid komma fram till den här matrisen, så den är unik. 

Vi har tre stycken ledande ettor i matrisens radekvivalenta matris på trappstegsform, alltså är matrisens rang = 3. Se definition \ref{matrisrang}.

\end{solution}
\end{document}