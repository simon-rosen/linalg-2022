\documentclass[../../main.tex]{subfiles}
\begin{document}
\begin{solution}{10}

Vi skriver om ekvationssystemet på matrisform och löser med gaussning (se sektion \ref{gaussjordan})

$$
\begin{pmatrix}[ccc|c]
2 & 3 & 0 & 0\\
2 & 3 & 4 & 5\\
-4& -6 & -4 & -5
\end{pmatrix}
\sim 
\begin{pmatrix}[ccc|c]
1 & \frac{3}{2} & 0 & 0\\
0 & 0 & 1 & \frac{5}{4}\\
0 & 0 & 0 & 0
\end{pmatrix}
$$

Det finns alltså \underline{oändligt många} lösningar till det här ekvationssystemet. Vilka är det?

Parametriserar $y = t$ och får följande ekvationssystem som beskriver lösningarna

\begin{cases}
x = -\frac{3}{2}t\\
y = t\\
z = \frac{5}{4}
\end{cases}

Vilket kan skrivas på vektorform som

$$
\colvec{x\\y\\z} = \colvec{-\frac{3}{2}t\\t\\\frac{5}{4}} = \colvec{0\\0\\\frac{5}{4}} + t\cdot\colvec{-\frac{3}{2}\\1\\0}
$$

Lösningen är alltså en linje.

\end{solution}
\end{document}