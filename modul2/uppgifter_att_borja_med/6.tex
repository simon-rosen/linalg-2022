\documentclass[../../main.tex]{subfiles}
\begin{document}
\begin{solution}{6}

\subsolution{a}
Vi söker någon matris $A$ och någon vektor $\Vec{x}$ så att

$$A\Vec{x}=a\cdot(1, 0, 2) + b\cdot(0, 2, -2) + c\cdot(3, -2, 3)$$

Detta uppfylls om

$$
A = \begin{pmatrix}[ccc]
1 & 0 & 3 \\
0 & 2 & -2\\
2 & -2 & 3\\
\end{pmatrix}, \text{ och } \Vec{x} = 
\begin{pmatrix}
a\\b\\c
\end{pmatrix}
$$

(Utför multiplikationen och kontrollera att  $A\Vec{x}=a\cdot(1, 0, 2) + b\cdot(0, 2, -2) + c\cdot(3, -2, 3)$ uppfylls om det inte känns självklart att vi valde dessa värden på $A$ och $\Vec{c}$)

\subsolution{b}
Jag skriver först om vektorekvationen lite för att vara övertydlig varför ekvationssystemet blir som det blir

$$
a\begin{pmatrix}
1\\0\\2
\end{pmatrix} + b \begin{pmatrix}
0\\2\\-2
\end{pmatrix} + c \begin{pmatrix}
3\\-2\\3
\end{pmatrix} = \begin{pmatrix}
1\cdot a + 0\cdot b + 3\cdot c\\
0\cdot a + 2\cdot b - 2\cdot c\\
2\cdot a - 2\cdot b + 3\cdot c\\
\end{pmatrix} = \begin{pmatrix}
0\\0\\0
\end{pmatrix}
$$

Ekvationssystemet blir alltså 
$$\begin{cases}
a + 3c = 0\\
2b - 2c = 0\\
2a - 2b + 3c = 0\\
\end{cases}$$

Nu skriver vi om det som en totalmatris och löser med gaussning (se sektion \ref{gaussjordan})

$$
\begin{pmatrix}[ccc|c]
1 & 0 & 3 & 0\\
0 & 2 & -2 & 0\\
2 & -2 & 3 & 0\\
\end{pmatrix}
\sim 
\begin{pmatrix}
r_1\\
\frac{1}{2}\cdot r_2\\
r_3 - 2r_1\\
\end{pmatrix}
\sim
\begin{pmatrix}[ccc|c]
1 & 0 & 3 & 0\\
0 & 1 & -1 & 0\\
0 & -2 & -3 & 0\\
\end{pmatrix}
\sim
\begin{pmatrix}
r_1\\r_2\\r_3+2r_2
\end{pmatrix}
\sim
$$
$
\begin{pmatrix}[ccc|c]
1 & 0 & 3 & 0\\
0 & 1 & -1 & 0\\
0 & 0 & -5 & 0\\
\end{pmatrix}
$

Man kan redan här se att vi kommer att komma fram till matrisen
$$\begin{pmatrix}[ccc|c]
1 & 0 & 0 & 0\\
0 & 1 & 0 & 0\\
0 & 0 & 1 & 0\\
\end{pmatrix}$$

Detta betyder att ekvationssystemet har lösningen $a = b = c = 0$, a.k.a. \underline{den triviala lösningen}.

\end{solution}
\end{document}