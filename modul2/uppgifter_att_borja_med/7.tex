\documentclass[../../main.tex]{subfiles}
\begin{document}
\begin{solution}{7}

Ekvationssystemet

$$
\begin{cases}
x + y + z = 1\\
2x + 3y + 4z = 12\\
-x + y + z = 7
\end{cases}
$$

Kan skrivas på matrisform, och sedan lösas med 'gaussning' (se sektion \ref{gaussjordan})

$$
\begin{pmatrix}[ccc|c]
  1 & 1 & 1 & 1\\
  2 & 3 & 4 & 12\\
  -1 & 1 & 1 & 7\\
\end{pmatrix} 
\sim 
\begin{pmatrix}[c]
  r_1\\
  r_2 - 2\cdot r_1\\
  r_3 + r_1\\
\end{pmatrix}
\sim 
\begin{pmatrix}[ccc|c]
  1 & 1 & 1 & 1\\
  0 & 1 & 2 & 10\\
  0 & 2 & 2 & 8\\
\end{pmatrix} 
\sim 
\begin{pmatrix}[c]
  r_1 - r_2\\
  r_2\\
  r_3 - 2\cdot r_2\\
\end{pmatrix}
\sim 
$$
% ny rad
$$
\begin{pmatrix}[ccc|c]
  1 & 0 & -1 & -9\\
  0 & 1 & 2 & 10\\
  0 & 0 & -2 & -12\\
\end{pmatrix} 
\sim 
\begin{pmatrix}[c]
  r_1\\
  r_2\\
  -\frac{1}{2} \cdot r_3\\
\end{pmatrix}
\sim 
\begin{pmatrix}[ccc|c]
  1 & 0 & -1 & -9\\
  0 & 1 & 2 & 10\\
  0 & 0 & 1 & 6\\
\end{pmatrix} 
\sim 
\begin{pmatrix}[c]
  r_1 + r_3\\
  r_2 - 2\cdot r_3\\
  r_3\\
\end{pmatrix}
\sim 
$$
% ny rad
$
\begin{pmatrix}[ccc|c]
  1 & 0 & 0 & -3\\
  0 & 1 & 0 & -2\\
  0 & 0 & 1 & 6\\
\end{pmatrix} 
$

Detta betyder att en lösning till ekvationssystemet är $x = -3, y = -2, z = 6$. Notera att det finns exakt en lösning. En geometrisk tolkning av ekvationssystemet är som skärningspunkten mellan tre plan, och om en sådan finns är det en punkt.

\end{solution}
\end{document}