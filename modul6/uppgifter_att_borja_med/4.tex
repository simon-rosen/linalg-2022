\documentclass[../../main.tex]{subfiles}
\begin{document}
\begin{solution}{4}

\subsolution{a}

Vi börjar med att bestämma egenvärden till $A$.

$$
\begin{vmatrix}
    1 - \lambda & 1\\
    0 & 1 - \lambda
\end{vmatrix} = 0 \iff
(1-\lambda)^2 = 0
$$

Man ser direkt att lösningarna till den karaktäriska ekvationen är $\lambda = 1$. Vi har alltså bara ett egenvärde till $A$ (och det egenvärdet har algebraisk multiplicitet 2 eftersom att det är en dubbelrot).

Nu använder vi detta egenvärde för att bestämma egenvektorer till $A$

$$
\begin{pmatrix}
    1 - 1 & 1\\
    0 & 1 - 1\\
\end{pmatrix}\Vec{v} = 
\begin{pmatrix}
    0 & 1\\
    0 & 0\\
\end{pmatrix}\Vec{v} = \Vec{0}
$$

Detta är samma sak som om vi hade gaussat och kommit fram till totalmatrisen

$$
\begin{pmatrix}[cc|c]
    0 & 1 & 0\\
    0 & 0 & 0
\end{pmatrix}
$$

Vad betyder detta? Jo om vi läser första raden ser vi att, $0x + 1y = 0$ vilket betyder att x kan vara vad som helst men y måste vara 0. Detta betyder att egenvektorer som hör till egenvärdet 1 tillhör

$$span\{\colvec{1\\0}\}$$

\subsolution{b}
En bas för $\Rn{2}$ kräver ju 2 linjärt oberoende vektorer, och vi har bara en linjärt oberoende egenvektor, så det går inte i det här fallet.

\subsolution{c}
En diagonalisering kräver en inverterbar matris $P$ med egenvektorerna som kolumner, och eftersom att vi bara har en egenvektor kommer vi inte kunna skapa en sådan.

\end{solution}
\end{document}