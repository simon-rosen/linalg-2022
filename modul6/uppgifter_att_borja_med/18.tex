\documentclass[../../main.tex]{subfiles}
\begin{document}
\begin{solution}{18}

(I den här modulen är det viktigt att man förstår egenvektorer osv. Uppgift 18 och 19 är inte så relaterade till det. Kvadratiska former återkommer dock i flervarren så det här blir inte sista gången ni ser dom.)

\subsolution{a}

$Q(x, y) = x^2 + 6xy + 4y^2 = xx + 6xy + 4yy$ är en kvadratisk form eftersom att varje term har grad 2. När vi skriver den på matrisform vill vi ha en matris så att ekvationen nedan uppfylls

$$
\begin{pmatrix}
x & y
\end{pmatrix} \begin{pmatrix}
a_{11} & a_{12}\\
a_{21} & a_{22}
\end{pmatrix} \colvec{x\\y}
= x^2 + 6xy + y^2
$$

(Notera att matrismultiplikationen fungerar eftersom att vi har en kolonnvektor till höger och en radvektor till vänster)

Vi kan utveckla uttrycket lite så blir det enklare att se varför svaret blir som det blir

$$
\begin{pmatrix}
x & y
\end{pmatrix} \begin{pmatrix}
a_{11} & a_{12}\\
a_{21} & a_{22}
\end{pmatrix} \colvec{x\\y}=
\begin{pmatrix}
x & y
\end{pmatrix} \colvec{a_{11}x + a_{12}y\\a_{21}x + a_{22}y} =
a_{11}x^2 + a_{12}xy + a_{21}xy + a_{22}y^2
$$
%ny rad
$
= x^2 + 6xy + 4y^2
$

Från detta kan vi direkt bestämma värdena på konstanterna $a_{11} = 1$ och $a_{22} = 4$. Vi ser även att $a_{12}xy + a_{21}xy = 6xy$, och eftersom att matriser som beskriver kvadratiska former ska vara symmetriska kan vi även bestämma  $a_{12} = a_{21} = 3$. 

\svar{Matrisen som är associerad till $Q$ är $\begin{pmatrix}
1 & 3\\
3 & 4
\end{pmatrix}$}

\subsolution{b}

Först lite om begreppen
\begin{itemize}
    \item \textit{Positivt (semi-)definit} betyder att $Q(x, y) \geq 0$ för alla värden på $x$ och $y$.
    \item \textit{Indefinit} betyder att $Q(x, y)$ antar både positiva och negativa värden när $x$ och $y$ varierar.
    \item \textit{Negativt (semi-)definit} betyder att $Q(x, y) \leq 0$ för alla värden på $x$ och $y$.
\end{itemize}

(positivt definit är egentligen $Q(x, y) > 0$ (strikt större), och positivt semidefinit är $Q(x, y) \geq 0$)

Vi kan undersöka $Q(x, y) = x^2 + 6xy + 4y^2$. Termerna $x^2$ och $y^2$ Kommer bara kunna anta positiva värden (eller 0) men termen $6xy$ kommer kunna anta både positiva och negativa värden. Några exempel:
\begin{itemize}
    \item $Q(1,1) = 1^2 + 6\cdot1\cdot1 + 4\cdot 1^2 = 11 > 0$
    \item $Q(1,-1) = 1^2 + 6\cdot1\cdot(-1) + 4\cdot 1^2 = -1 < 0$
\end{itemize}

Alltså, $Q(x, y)$ kan anta både positiva och negativa värden och är därav \textit{indefinit}.

\end{solution}
\end{document}