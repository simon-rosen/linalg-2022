\documentclass[../../main.tex]{subfiles}
\begin{document}
\begin{solution}{13}

Vi börjar med att beräkna egenvärdena till $A$

$$
\begin{vmatrix}
    1 - \lambda & 1 & 1\\
    1 & 1 - \lambda & 1 \\
    1 & 1 & 1 - \lambda
\end{vmatrix} = \{\text{kofaktorutvecklar längs första kolummnen}\} = $$
%ny rad
$
(1 - \lambda) \begin{vmatrix}
    1 - \lambda & 1\\
    1 & 1 - \lambda
\end{vmatrix} - 1\begin{vmatrix}
    1 & 1\\
    1 & 1 - \lambda
\end{vmatrix} + 
1\begin{vmatrix}
    1 & 1\\
    1 - \lambda & 1
\end{vmatrix}$
% ny rad
\\\\
$ 
= (1 - \lambda)[(1 - \lambda)^2 - 1] - 1[(1- \lambda) - 1] + 1[1 - (1 - \lambda)]
= \text{\{... lång uträkning ...\}} = -\lambda ^3 + 3 \lambda^2 = \lambda ^2 (3 - \lambda) = 0
$

Vi ser att lösningar till den karaktäristiska ekvationen är 

\begin{cases}
    \lambda = 0 \text{ (dubbelrot, alltså algebraisk multiplicitet 2)}\\
    \lambda = 3 \text{ (enkelrot, alltså algebraisk multiplicitet 1)}
\end{cases}

Den geometriska multipliciteten för ett egenvärde är antalet egenvektorer som hör till just det egenvärdet. Vi undersöker

\underline{$\lambda = 0$}

$
\begin{pmatrix}
    1 - 0 & 1 & 1\\
    1 & 1 - 0 & 1\\
    1 & 1 & 1 - 0
\end{pmatrix} \Vec{v} = \Vec{0}
$

Vi löser detta med gaussning

$$
\begin{pmatrix}[ccc|c]
    1 & 1 & 1 & 0\\
    1 & 1 & 1 & 0\\
    1 & 1 & 1 & 0\\
\end{pmatrix} \skipgauss 
\begin{pmatrix}
    1 & 1 & 1 & 0\\
    0 & 0 & 0 & 0\\
    0 & 0 & 0 & 0\\
\end{pmatrix}
$$

Alltså

\begin{cases}
    x = - s - t\\
    y = s\\
    z = t
\end{cases}

Där $s, t \in \Rn{}$

Detta betyder att egenrummet (spannet av de linjärt oberoende egenvektorerna till detta egenvärde) är 

$$
span\{\colvec{-1\\1\\0}, \colvec{-1\\0\\1}\}
$$

Det består alltså av 2 vektorer, så den geometriska multipliciteten för detta egenvärde är 2.

\underline{$\lambda = 3$}

$\begin{pmatrix}
    1 - 3 & 1 & 1\\
    1 & 1 - 3 & 1 \\
    1 & 1 & 1 - 3
\end{pmatrix}\Vec{v} =
\begin{pmatrix}
    -2 & 1 & 1\\
    1 & -2 & 1\\
    1 & 1 & -2
\end{pmatrix} \Vec{v} =
\Vec{0}$

Vi löser detta med gaussning

$$
\begin{pmatrix}[ccc|c]
-2 & 1 & 1 & 0\\
1 & -2 & 1 & 0\\
1 & 1 & -2 & 0\\
\end{pmatrix} \skipgauss
\begin{pmatrix}
1 & 0 & -1 & 0\\
0 & 1 & -1 & 0\\
0 & 0 & 0 & 0
\end{pmatrix}
$$

Man kan lösa det som innan och ta fram egenrummet till $\lambda = 0$, men man kan också se direkt på lösningsmängden att det innehåller \underline{en} fri variabel och kommer bilda en linje, vilket även innebär att den geometriska multipliciteten för detta egenvärde är 1.

\svar{Egenvärdet 0 har algebraisk och geometrisk multiplicitet 2, och egenvärdet 3 har algebraisk och geometrisk multiplicitet 1.}

\end{solution}
\end{document}