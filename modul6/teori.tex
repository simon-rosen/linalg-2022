\documentclass[../main.tex]{subfiles}
\begin{document}
\begin{definition}{Egenvektor}
En egenvektor $\Vec{v}$ för en matris $A$ är en vektor som bevarar sin riktning under en linjär avbildning med avbildningsmatris $A$. Detta skrivs

$$
T(\Vec{v}) = A\Vec{v} = \lambda \cdot \Vec{v}
$$

Alltså, det enda som händer med $\Vec{v}$ under avbildningen är att den skalas om med någon skalfaktor $\lambda$.
\end{definition}

\begin{definition}{Egenvärde}
Ett egenvärde $\lambda$ för en matris $A$ är skalfaktorn hos för någon av dessa \underline{egenvektorer} till A.
\end{definition}

\subsection{Metod för att ta fram egenvärden och egenvektorer till en matris}
\subsubsection{ta fram egenvärden}
När man vill ta fram egenvektorer till en matris $A$ börjar man alltid med att ta fram egenvärdena. Dessa fås genom att man löser matrisens \underline{karaktäristiska ekvation}

$$
det(A - \lambda E) = 0
$$

Där $\lambda$ står för ett egenvärde till matrisen och $E är enhetsmatrisen$. Vi kan utveckla detta lite. Låt oss ta en exempelmatris

$$
A = \begin{pmatrix}
1 & 2\\
1& 2
\end{pmatrix}
$$

Då får vi

$$
det(A - \lambda E) = 
det(\begin{pmatrix}
1 & 2\\
1 & 2
\end{pmatrix} -
\lambda\begin{pmatrix}
1&0\\0&1
\end{pmatrix}) = 
det(
\begin{pmatrix}
1 - \lambda & 2\\
1 & 2 - \lambda
\end{pmatrix}
)
=
\begin{vmatrix}
1-\lambda & 2\\
1 & 2 - \lambda
\end{vmatrix} = 0
$$

Vi beräknar determinanten som vanligt och får ett polynom

\begin{align*}
(1-\lambda)(2-\lambda) - 2 = 0\\
2 - \lambda - 2\lambda + \lambda^2 - 2 = 0\\
\lambda^2 - 3\lambda = 0\\
\lambda (\lambda - 3) = 0
\end{align*}

Som har lösningarna $\lambda = 0$ och $\lambda = 3$. Det här betyder att matrisen har egenvärdena 0 och 3.

\subsection{Ta fram egenvektorer}
Man kan ta fram egenvektorer till en matris när man har tagit fram dess egenvärden. Man gör detta genom att lösa ekvationen

$$
(A - \lambda E) = \Vec{0}
$$

Där man sätter in ett av de egenvärden som man har tagit fram som $\lambda$ i ekvationen. Resultatet kommer bli de egenvektorer som hör till det egenvärdet. Dessa vektorer kommer under en linjär avbildning med $A$ som avbildningsmatris att behålla sin riktning, och de kommer att skalas om med skalfaktorn $\lambda$.

Låt oss fortsätta med exemplet ovan. Vi har beräknat egenvärdena till vår exempelmatris $A = \begin{pmatrix}
    1 & 2\\
    1 & 2
\end{pmatrix}$ till 0 och 3. Vi får nu använda dessa egenvärden för att ta fram de egenvektorer som hör till dessa

\underline{\textbf{fallet när }$\lambda = 0$}\\

\begin{align*}
    (A - \lambda E) = \vec{0}\\
    \begin{pmatrix}
        1 & 2\\
        1 & 2
    \end{pmatrix} - 
    \begin{pmatrix}
        0\lambda & 0\\
        0 & 0\lambda
    \end{pmatrix} = \vec{0}\\
    \begin{pmatrix}
        1 & 2\\
        1 & 2
    \end{pmatrix} = \Vec{0}
\end{align*}

Det här kan vi lösa med gaussning

$$
\begin{pmatrix}[cc|c]
    1 & 2 & 0\\
    1 & 2 & 0\\
\end{pmatrix} \skipgauss
\begin{pmatrix}
    1 & 2 & 0\\
    0 & 0 & 0
\end{pmatrix}
$$

Lösningsmängden till det här ekvationssystemet blir 

$$
\colvec{x\\y} = t\colvec{-2\\1}, t\in\Rn{}
$$

Detta bildar en linje i $\Rn{2}$, och alla vektorer som tillhör den är egenvektorer till $A$ med egenvärdet 0.


\underline{\textbf{fallet när }$\lambda = 0$}\\

Man gör på samma sätt när man vill beräkna de egenvektorer som hör till egenvärdet 3. 

När man har beräknat de egenvektorer som hör till respektive egenvärde är man klar, då har man tagit fram matrisens egenvektorer. 

\subsection{Diagonalisering}

\end{document}