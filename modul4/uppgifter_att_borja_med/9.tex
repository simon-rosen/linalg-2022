\documentclass[../../main.tex]{subfiles}
\begin{document}
\begin{solution}{9}

När vektorer $\Vec{u} = (u_1, u_2, u_3)$ projiceras på $x_1x_2$-planet kommer det resultera i vektorer på formen = $(u_1, u_2, 0)$. Alltså kommer dessa vektorer ha $x_3=0$, medans $u_1$ och $u_2$ är oförändrade. Den avbildningsmatris som åstadkommer detta är

$$\begin{pmatrix}
1&0&0\\0&1&0\\0&0&0
\end{pmatrix}$$

Vi testar med en vektor $\Vec{v} = (1, 1, 1) \in \Rn{3}$. Projektionen av denna vektor på $x_1x_2$-planet borde vara vektorn $(1, 1, 0)$

$$
\begin{pmatrix}
1&0&0\\0&1&0\\0&0&0
\end{pmatrix} \colvec{1\\1\\1\\} =
\colvec{1+0+0\\0+1+0\\0+0+0} = \colvec{1\\1\\0}
$$

I den här uppgiften kan man se vad avbildningsmatrisen ska vara utan att göra några avancerade beräkningar.

\end{solution}
\end{document}