\documentclass[../../main.tex]{subfiles}
\begin{document}
\begin{solution}{13}

Vi reducerar matrisen med gaussning

$$
A = \begin{pmatrix}
1&2&3&0\\
5&2&0&1\\
1&1&2&1
\end{pmatrix}
\skipgauss
\begin{pmatrix}
1&0&0&1\\
0&1&0&-2\\
0&0&1&1
\end{pmatrix}
$$

Detta betyder att de första kolonnerna är linjärt oberoende och att de utgör bildrummet till den linjära avbildningen som ges av matrisen $A$. Alltså

$$span(\{ \colvec{1\\5\\1}, \colvec{2\\2\\1}, \colvec{3\\0\\2} \})$$

Nollrummet tas fram genom att vi parametriserar lösningsmängden till 

$$A\Vec{x} = \colvec{0\\0\\0}$$

I den sista kolonnen har vi en fri variabel, lösningsmängden till den här ekvationen, alltså nollrummet till vår linjära avbildning, blir alltså 

$$span(\{\colvec{-1\\2\\-1\\1}\})$$ 


Jag såg på matrisen att det här är svaret, men om man inte tycker att det är uppenbart än kan börja med att parametrisera lösningsmängden (repetera uppgifterna från modul 3 om det är svårt)

\begin{cases}
x + w = 0\\
y - 2w = 0\\
z + w = 0\\
w = t
\end{cases}

\end{solution}
\end{document}