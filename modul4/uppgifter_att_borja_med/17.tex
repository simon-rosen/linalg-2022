\documentclass[../../main.tex]{subfiles}
\begin{document}
\begin{solution}{17} % ändra till rätt uppgiftsnummer

    Vi börjar med gaussning på matrisen.

    $$
        \begin{pmatrix}[cccc]
            1  & 1 & 0 & 1 \\
            1  & 0 & 1 & 2 \\
            -1 & 1 & 1 & 0
        \end{pmatrix}
        \skipgauss
        \begin{pmatrix}[cccc]
            1 & 0 & 0 & 1 \\
            0 & 1 & 0 & 0 \\
            0 & 0 & 1 & 1
        \end{pmatrix}
    $$


    De tre första kolonnerna är linjärt oberoende och utgör alltså bildrummet. Alltså är bildrummet

    $$
        span({\begin{pmatrix}[c]
                    1  \\
                    1  \\
                    -1 \\
                \end{pmatrix}
                \begin{pmatrix}[c]
                    1 \\
                    0 \\
                    1 \\
                \end{pmatrix}
                \begin{pmatrix}[c]
                    0 \\
                    1 \\
                    1 \\
                \end{pmatrix}
            })
    $$

    För att bestämma nollrummet parametriserar vi lösningsmängden skriver vi på formen $Ax = \vec{0}$.

    $$
        \begin{pmatrix}[cccc]
            1 & 0 & 0 & 1 \\
            0 & 1 & 0 & 0 \\
            0 & 0 & 1 & 1
        \end{pmatrix}
        \begin{pmatrix}[c]
            x_1 \\
            x_2 \\
            x_3
        \end{pmatrix}
        =
        \begin{pmatrix}[c]
            1 \\
            0 \\
            1
        \end{pmatrix}
    $$

    Nollrummet till matrisen är alltså $span({\begin{pmatrix}[c]
                    -1 \\
                    0  \\
                    -1 \\
                    1
                \end{pmatrix}})$.





\end{solution}
\end{document}
