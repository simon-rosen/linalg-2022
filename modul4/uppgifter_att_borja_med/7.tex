\documentclass[../../main.tex]{subfiles}
\begin{document}
\begin{solution}{7}

Vi vill ha en linjär avbildning som speglar alla vektorer $\vec{v} \in \Rn{2}$ i x-axeln. 

\begin{vectors2d}{-3}{-3}{3}{3}
    \draw[line width=1pt,-stealth](0,0)--(2,2) node[anchor=south east]{$\vec{v}$};
    \draw[line width=1pt,gray!60,-stealth](0,0)--(2,-2) node[anchor=north west]{speglad $\vec{v}$};
\end{vectors2d}

Alltså söker vi en avbildningsmatris som gör så att vektorer behåller samma x-värde men multiplicerar sitt y-värde med $-1$. Denna matris är
$$
\begin{pmatrix}
1&0\\0&-1
\end{pmatrix}
$$

Vi kan testa att multiplicera $\vec{v} = \colvec{1\\1}$ och se om vi får fram $\colvec{1\\-1}$ efter matrismultiplikationen

$$
\begin{pmatrix}
1&0\\0&-1
\end{pmatrix} \colvec{1\\1} = \colvec{1+0\\0+(-1)} = \colvec{1\\-1}
$$

I den här uppgiften kan man se vad avbildningsmatrisen ska vara utan att göra några beräkningar. 
\end{solution}
\end{document}