\documentclass[../main.tex]{subfiles}
\begin{document}
\subsection{Linjära avbildningar}
Linjära avbildningar är egentligen bara matrismultiplikation mellan en matris $A$ (som kallas \underline{avbildningsmatris}) och en vektor $\Vec{u}$. En sådan multiplikation kommer ge tillbaka en ny vektor $\Vec{v}$. Vi säger att vektorn \underline{avbildas} på en annan vektor.

$$A\Vec{u} = \Vec{v}$$

$\Vec{u}$ och $\Vec{v}$ behöver inte nödvändigtvis vara lika stora, en avbildning kan alltså gå från en dimension till en annan. Ett exempel på detta är kameror: De tar något som händer i vår värld (3D) och gör om det till ett fotografi (2D).

\subsection{Bildrum / Kolonnrum (col)}


\subsection{Nollrum / Kärna (kernel)}


\end{document}