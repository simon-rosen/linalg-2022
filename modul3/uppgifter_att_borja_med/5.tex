\documentclass[../../main.tex]{subfiles}
\begin{document}
\begin{solution}{5}

Kom ihåg att ett vektorrum ska uppfylla följande
\begin{itemize}
    \item Det ska vara \underline{slutet under skalärmultiplikation}. Dvs för varje vektor som tillhör vektorrumet så ska man kunna multiplicera den med en skalär och få en ny vektor som också ligger i rummet, om man skulle få en vektor som inte tillhör rummet så är det inte ett vektorrum.
    \item Det ska vara \underline{slutet under vektoraddition}. Dvs för alla par av vektorer $\Vec{u}$ och $\Vec{v}$ som tillhör vårt rum så måste $\Vec{w} = \Vec{u} + \Vec{v}$ också tillhöra vårt rum, annars är det inget vektorrum.
    \item Vektorrummet måste innehålla nollvektorn $\vec{0}$.
\end{itemize}

Ett delrum till $\Rn{2}$ ska uppfylla dessa krav och dessutom ha en dimension $\leq 2$. Det finns många val på mängder av vektorer som inte är giltiga delrum till $\Rn{2}$. Här är några exempel

\begin{itemize}
    \item linjer som inte går igenom origo, eftersom att de då inte innehåller nollvektorn.
    \item Ändliga mängder av vektorer, eftersom att man då kommer kunna skapa nya vektorer genom vektoraddition och multiplikation med skalärer som inte ingår i den ändliga mängden av vektorer. Det finns ett specialfall av en ändlig mängd vektorer som faktiskt är ett giltigt delrum till alla $\Rn{n}$: $\{\vec{0}\}$, dvs mängden av vektorer som endast innehåller nollvektorn (den uppfyller alla krav som ställs på delrum).
\end{itemize}

\end{solution}
\end{document}