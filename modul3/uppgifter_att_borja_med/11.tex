\documentclass[../../main.tex]{subfiles}
\begin{document}
\begin{solution}{11}

\subsolution{a}

$$
\det\begin{pmatrix}[ccc]
1&0&2\\
2&1&1\\
0&1&2
\end{pmatrix} 
=
\begin{vmatrix}
1&0&2\\
2&1&1\\
0&1&2
\end{vmatrix}
$$

Jag väljer att kofaktorutveckla längs första raden

$$
\begin{vmatrix}
1&0&2\\
2&1&1\\
0&1&2
\end{vmatrix}
= 1\cdot \begin{vmatrix}[cc]
1& 1\\ 
1&2
\end{vmatrix}
- 0\cdot \begin{vmatrix}
2 & 1\\
0&2
\end{vmatrix}
+ 2\cdot \begin{vmatrix}
2&1\\
0&1
\end{vmatrix}
= 1(2-1) - 0(4-0) + 2(2 - 0) = 1 + 4 = 5
$$

\subsolution{b}
Eftersom att $\det B \neq 0$ så vet vi att matrisen är inverterbar.

\subsolution{c}
Nej, det har ingen icke-trivial lösning. Anledningen är att $\det B = 0 \iff $ Ekvationssystemet har en unik lösning, och eftersom att vi vet att den triviala lösningen är en lösning så är det den enda lösningen som finns.

\subsolution{d}
Vi kan börja med att ta fram $B^T$ (B:s \underline{transponat})

$$B^T = \begin{pmatrix}
1&2&0\\
0&1&1\\
2&1&2
\end{pmatrix}$$

Man byter alltså så att kolumner blir rader och vice versa. Låt oss nu beräkna $\det B^T$

$$
\det B^T = \begin{vmatrix}
1&2&0\\
0&1&1\\
2&1&2
\end{vmatrix}
$$

Jag väljer nu att kofaktorutveckla längs den första kolumnen

$$
\det B^T = \begin{vmatrix}
1&2&0\\
0&1&1\\
2&1&2
\end{vmatrix}
= 1\cdot \begin{vmatrix}[cc]
1& 1\\ 
1&2
\end{vmatrix}
- 0\cdot \begin{vmatrix}
2 & 0\\
1 & 2
\end{vmatrix}
+ 2\cdot \begin{vmatrix}
2&0\\
1&1
\end{vmatrix}
= 1(2-1) - 0(4-0) + 2(2 - 0) = 1 + 4 = 5
$$

Alltså $\det B = \det B^T$. 

\end{solution}
\end{document}