\documentclass[../../main.tex]{subfiles}
\begin{document}
\begin{solution}{17}

Den första linjen har riktningsvektorn $\colvec{1\\-1\\1}$ och den andra har $\colvec{0\\2\\1}$. Vi vill ta fram kryssprodukten av dessa för att kunna skriva planet på standardform

$$
\colvec{1\\-1\\1} \times \colvec{0\\2\\1} = \colvec{-1-2\\0-1\\2-0} = \colvec{-3\\-1\\2}
$$

Båda linjerna innehåller punkten $\Vec{0} = (0,0,0)$. Vi sätter alltså in den punkten i ekvationen

\begin{align*}
-3(x - x_0) -1(y - y_0) +2(z-z_0) = 0\\
-3(x-0) -1(y-0) +2(z-0) = 0\\
-3x -y +2z = 0
\end{align*}

Svaret är alltså $-3x -y +2z = 0$. I facit skriver de att svaret är $3x + y - 2z = 0$, detta är också giltigt och man får det om man byter plats på vektorerna vi tog kryssprodukten av.

\end{solution}
\end{document}