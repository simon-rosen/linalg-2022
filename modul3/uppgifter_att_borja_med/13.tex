\documentclass[../../main.tex]{subfiles}
\begin{document}
\begin{solution}{13}

Vi har tre punkter som ligger i planet. Låt oss kalla dom för
$$
\begin{cases}
A = (1, 0, 1)\\
B = (2, 1, 1)\\
C = (-1, 3, 5)
\end{cases}
$$

Med dessa punkter kan vi ta fram två riktningsvektorer som är parallella med planet


$$
\Vec{AB} = \Vec{OB} - \Vec{OA} = \colvec{2\\1\\1} - \colvec{1\\0\\1} = \colvec{1\\1\\0}
$$
$$
\Vec{AC} = \Vec{OC} - \Vec{OA} = \colvec{-1\\3\\5} - \colvec{1\\0\\1} = \colvec{-2\\3\\4}
$$

För att ta fram planets standardekvation behöver vi en vektor som är ortogonal mot planet. Denna fås genom att ta kryssprodukten av de två vektorerna som är parallella med planet

$$
\Vec{AB} \times \Vec{AC} = \colvec{1\\1\\0} \times \colvec{-2\\3\\4} = \colvec{4-0\\0-4\\3-(-2)} = \colvec{4\\-4\\5}
$$


Nu behöver vi bara sätta in en punkt i ekvationen 

$$
4(x-x_0) - 4(y - y_0) + 5(z-z_0) = 0
$$

Jag väljer punkten $A$

\begin{align*}
4(x - 1) - 4(y - 0) + 5(z - 1) = 0\\
4x - 4 - 4y + 5z - 5 = 0\\
4x - 4y + 5z = 9
\end{align*}

\end{solution}
\end{document}