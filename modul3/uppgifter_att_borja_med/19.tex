\documentclass[../../main.tex]{subfiles}
\begin{document}
\begin{solution}{19}

Man skulle kunna beräkna matrisen som ges av $B^{-1}AB$ och sedan ta determinanten av denna. Men vi kan använda några räkneregler för determinanter som gör denna uppgift mycket enklare (se föreläsning 6)

$$
\begin{cases}
\det AB = \det A \cdot \det B\\
\det A^{-1} = \frac{1}{\det A} 
\end{cases}
$$

Notera att för att kunna använda den andra regeln så måste $A$ vara en inverterbar matris. Men här kan vi anta att $B$ är inverterbar för annars skulle det inte gå att beräkna $\det B^{-1}AB$. Låt oss utveckla uttrycket med hjälp av räknereglerna

$$
\det B^{-1}AB = \det B^{-1} \cdot \det A \cdot \det B = \frac{1}{\det B} \cdot \det A \cdot \det B = \det A
$$

$\det B^{-1}$ och $\det B$ tog alltså ut varandra. Nu räcker det alltså att beräkna $\det A$

$$
\det B^{-1}AB = \det A = 
\begin{vmatrix}
1 & 2\\
2 & 6
\end{vmatrix} =
1\cdot6 - 2\cdot2 = 6-4 = 2
$$

\svar{$\det B^{-1}AB = 2$}

\end{solution}
\end{document}