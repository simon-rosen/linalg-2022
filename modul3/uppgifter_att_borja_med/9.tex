\documentclass[../../main.tex]{subfiles}
\begin{document}
\begin{solution}{9}

Nej, det är inte sant. Här är ett exempel

\begin{vectors2d}{0}{0}{5}{5}
\draw[line width=1pt,blue!60,-stealth](0,0)--(3,0) node[anchor=south west]{$\vec{v}$};
\draw[line width=1pt,blue!60,-stealth](0,0)--(0,3) node[anchor=south west]{$\vec{u}$};
\draw[line width=1pt,blue!60,-stealth](0,0)--(3,3) node[anchor=south west]{$\vec{w}=\vec{u} + \vec{v}$};
\end{vectors2d}

Det är uppenbart att de är linjärt beroende eftersom att $\vec{w} = \vec{u} + \vec{v}$, och man ser också att inte några av dom är parallella.

\end{solution}
\end{document}