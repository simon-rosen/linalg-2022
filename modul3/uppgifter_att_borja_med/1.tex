\documentclass[../../main.tex]{subfiles}
\begin{document}
\begin{solution}{1}

\subsolution{a}

Kom ihåg att vektorer är \underline{linjärt oberoende} om man inte kan skapa någon av dom genom en linjärkombination av de andra vektorerna. Om de skulle vara linjärt beroende skulle vi kunna uppfylla vektorekvationen

$$s\Vec{u} + t\Vec{v} = \Vec{w} \iff s\colvec{1\\2\\3\\} + t\colvec{3\\2\\1} = \colvec{1\\0\\0}$$

Låt oss testa detta med gaussning

$$
\begin{pmatrix}[cc|c]
1 & 3 & 1\\
2 & 2 & 0\\
3 & 1 & 0\\
\end{pmatrix}
\sim ... \sim 
\begin{pmatrix}[cc|c]
1 & 3 &           1 \\
0 & 1 & \frac{1}{2} \\
0 & 0 &           1 \\
\end{pmatrix}
$$

Det här systemet går alltså inte att lösa (det är inkonsistent), vilket betyder att vi inte kan skriva $\Vec{w}$ som en linjärkombination av $\Vec{u}$ och $\Vec{v}$, alltså är vektorerna linjärt oberoende.

\subsolution{b}
Nej, eftersom att de är linjärt oberoende kan vi inte skriva någon av dom som en linjärkombination av de andra.

\subsolution{c}
Ja, eftersom att de tre vektorerna är linjärt oberoende kommer de att spänna upp hela $\Rn{3}$. Och $\Rn{3}$ är ett delrum till $\Rn{3}$.

\end{solution}
\end{document}