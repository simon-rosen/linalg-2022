\documentclass[../main.tex]{subfiles}
\begin{document}
%% kryssprodukten
\begin{formel}{(kryssprodukt)}
\label{kryssprod} 
Kryssprodukten av två vektorer $\vec{u} = (u_1, u_2, u_3)$ och $\vec{v} = (v_1, v_2, v_3)$ är
\[\vec{u}\times \vec{v} = (u_2\cdot v_3 - u_3\cdot v_2, u_3\cdot v_1 - u_1\cdot v_3, u_1\cdot v_2 - u_2\cdot v_1)\]

En viktig egenskap kryssprodukten $\vec{u}\times\vec{v}$ är att den \underline{är en vektor} som är ortogonal mot båda vektorerna $\vec{u}$ och $\vec{v}$.
\end{formel}


%% kryssproduktens storlek
\begin{formel}
\label{kryssprodsize} 
Storleken av kryssprodukten av två vektorer $\vec{u}$ och $\vec{v}$ är
\[|\vec{u}\times \vec{v}| = |\vec{u}||\vec{v}|sin(\alpha)\]
Där $\alpha$ är vinkeln mellan $\vec{u}$ och $\vec{v}$.
\end{formel}

% delrum
\subsubsection{Vektorrum och delrum}
\label{vektorrum}
Ett rum ska uppfylla följande krav
\begin{itemize}
    \item Det ska vara slutet under
    \begin{itemize}
        \item Skalärmultiplikation - dvs om man vet att en vektor $\vec{v}$ ligger i rummet så ska även $t\cdot\vec{v}$ ligga i rummet.
        \item Vektoraddition med vektorer i rummet - dvs om man har två vektorer $\vec{u}$ och $\vec{v}$ i rummet så ska även $\vec{u}+\vec{v}$ ligga i rummet.
    \end{itemize}
    \item Det måste innehålla nollvektorn $\vec{0}$.
\end{itemize}

Ett delrum är lite som en delmängd till ett annat rum. Det måste uppfylla samma krav som rummet det är ett delrum till, men det kan ha lägre dimension. 

Exempel: $\Rn{2}$ är ett delrum till $\Rn{3}$. Ett till exempel (specialfall): $\Rn{3}$ är ett delrum till $\Rn{3}$.

\end{document}
