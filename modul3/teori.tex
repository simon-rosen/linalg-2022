\documentclass[../main.tex]{subfiles}
\begin{document}
%% kryssprodukten
\begin{formel}{(kryssprodukt)}
\label{kryssprod} 
Kryssprodukten av två vektorer $\vec{u} = (u_1, u_2, u_3)$ och $\vec{v} = (v_1, v_2, v_3)$ är
\[\vec{u}\times \vec{v} = (u_2\cdot v_3 - u_3\cdot v_2, u_2\cdot v_3 - u_3\cdot v_2, u_2\cdot v_3 - u_3\cdot v_2)\]

En viktig egenskap kryssprodukten $\vec{u}\times\vec{v}$ är att den \underline{är en vektor} som är ortogonal mot båda vektorerna $\vec{u}$ och $\vec{v}$.
\end{formel}


%% kryssproduktens storlek
\begin{formel}
\label{kryssprodsize} 
Storleken av kryssprodukten av två vektorer $\vec{u}$ och $\vec{v}$ är
\[|\vec{u}\times \vec{v}| = |\vec{u}||\vec{v}|sin(\alpha)\]
Där $\alpha$ är vinkeln mellan $\vec{u}$ och $\vec{v}$.
\end{formel}

\end{document}